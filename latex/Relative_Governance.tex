\documentclass[12pt]{article}
\usepackage[utf8]{inputenc}

\title{A governance valuation framework}
\author{@jbrukh}
\date{March 2019}

\begin{document}

\maketitle

\section{Voting distribution}

Suppose you are in a governance system where voting power is denoted by a token across $n$ token holders. Let the token holder stake be denoted by a vector

$$\mathbf{s} = (s_1,s_2,\ldots,s_n)$$\\
where $\sum s_j = 1$ and $s_j > 0$ for all $j=1,\ldots,n$. Now, define a \textit{vote} as an $n$-dimensional vector of binary digits

$$\mathbf{v} = \left(v_1, v_2, \ldots, v_n\eight)$$\\
where $v_j \in \{0, 1\}$. Given a real positive number $M\in(0,1]$, and some distribution $\mathbf{s}$, and some vote $\mathbf{v}$ such that 

$$ \mathbf{s}\cdot \mathbf{v} > M$$\\
we will say that ``the vote passes''. Otherwise, ``the vote is rejected.'' We say that $M$ is the ``majority threshold'' or ``majority required for the vote to pass''.

\section{Decisive votes}

For some token holders, in some votes, their stake and vote is the difference between a proposal passing or not passing. Let us formalize this as follows. Suppose $\mathbf{v}$ is some vote, and define $\mathbf{v_0}(j) := (v_1, v_2, \ldots, v_j, \ldots, v_n), v_j = 0$, the vote where the $j$-th digit is set to $0$. Similarly, define $\mathbf{v_1}(j) := (v_1, v_2, \ldots, v_j, \ldots, v_n), v_j = 1$. Given a stakeholder distribution $\mathbf{s}$ and a vote $\mathbf{v}$, we say that token holder $j$'s vote was \textit{decisive} if and only if the following condition holds.

\begin{equation}
        \mathbf{s}\cdot\mathbf{v_0}(j) \leq M\text{\ but\ } \mathbf{s}\cdot\mathbf{v_1}(j) > M
\end{equation}\\
Intuitively, a token holder's stake $s_j$ in a particular vote $\mathbf{v}$ is decisive if and only if it has the power to overturn the outcome of the vote.
Note that we can shorten the characterization of a decisive vote as follows. A token holder $j$ is decisive to a vote with respect to token distribution $\mathbf{s}$ if and only if

\begin{equation}
    M - s_j < \mathbf{s}\cdot\mathbf{v_0}(j) \leq M
\end{equation}\\
The last inequality follows by definition from $(1)$. The strict inequality follows by subtracting $s_j$ from both sides of the latter inequality in (1), noting that $\mathbf{s}\cdot\mathbf{v_1}(j) - s_j = \mathbf{s}\cdot\mathbf{v_0}(j)$.


\section{Calculating decisiveness}

Let $\mathcal{V}_n$ be set of all possible votes on $n$ token holders, that is, the set of all $n$-dimensional vectors of binary digits. Given some stake distribution $\mathbf{s}$, and given some token holder $j$, let $\mathcal{D}_n(j)$ be the set of decisive votes with respect to $j$. Then define the \textif{decisiveness of $j$ with respect to $\mathbf{s}$} as $$D(\mathbf{s}, j) := |\mathcal{D}_n(j)|/|\mathcal{V}_n|.$$
Note also that $|D(\mathbf{s}, j)|\leq 1$ and $|\mathcal{V}_n| = 2^n$.

\end{document}